\section*{Extracurriculars}

\begin{cventries}
  \cventry{EPFL Computer Graphics Competition}
  {Runner-up}
  {June 2021}
  {}
  {%
    \begin{cvitems}
    \item Implemented a path tracer with advanced features in \cvlanguage{C++} in order to render a realistic scene for the 3D rendering competition organized by the Computer Graphics course at EPFL\@.
      \item Voted as second most interesting/technically impressive scene among 70+ entrants, from a jury consisting of researchers from Disney, Weta Digital and Nvidia.
    \end{cvitems}
  }
  %cventry{Computer Graphics Competition}
  %{First place}
  %{June 2021}
  %{}
  %{%
  %  \begin{cvitems}
  %    \item Used \cvlanguage{C++} to implement a small 3D scene rendered with OpenGL using HDR rendering, shadow mapping, and vertex displacement.  \textbf{\githublink{isolation}{Project on GitHub.}}
  %    \item Voted best/most interesting real-time rendered scene among 20+ entrants for the real-time 3D rendering competition organized by the Computer Graphics course at the University of Groningen.
  %  \end{cvitems}
  %}
  \cventry{Student Association \textit{Cover}}
  {Hackathon Committee Member}
  {September 2018 --- March 2020}
  {}
  {%
    \begin{cvitems}
      \item Helped organize LAN parties and Hackathons in partnership with a local company, each with 40+ attendants.
      \item Selected and tested programming challenges for Hackathons.
      \item Designed the committee logo.
    \end{cvitems}
  }
\end{cventries}

\section*{Projects}
\begin{cventries}
  \cventry{Semester Project}
  {\textit{Dr.Jit Kernel Caching}}
  {}
  {}
  {%
    \begin{cvitems}
	\item Implemented a machine code caching layer for \textbf{\href{https://rgl.epfl.ch/publications/Jakob2022DrJit}{Dr.Jit \faExternalLink*}}, an experimental just-in-time compiler for physically based rendering developed by the Realistic Graphics Lab at EPFL, written in \cvlanguage{C++}.
		\item Devised and implemented a way to reuse previously-generated machine code of a rendering kernel while allowing changes to the scene's parameters (e.g. camera position, material albedo and reflectivity)
		\item Tested the feature on Dr.Jit's LLVM and CUDA backends.
    \end{cvitems}
  }
  %\cventry{Bachelor's Project}
  %{\textit{Arbitrary local refinement of gradient meshes}}
  %{}
  %{}
  %{%
  %  \begin{cvitems}
  %  \item Research component: devised a new way to split individual cells in gradient meshes, a popular vector graphics primitive.
  %  \item Development component: used \cvlanguage{C++} to build an interactive gradient mesh editor, allowing users to move and split cells using the above mentioned refinement scheme.
  %  \item Soon to be published in an academic journal.
  %  \end{cvitems}
  %}
  \cventry{Personal Project}
  {\textit{Oxide}}
  {}
  {}
  {%
    \begin{cvitems}
    \item Designed and implemented a small, general-purpose scripting language using \cvlanguage{Rust}.\textbf{\githublink{oxide}{Project on GitHub.}}
    \item Created a custom instruction set and virtual machine for program instructions to be compiled to and run on.
    \end{cvitems}
  }
  %\cventry{Personal Project}
  %{\textit{Prayer}}
  %{}
  %{}
  %{%
  %  \begin{cvitems}
  %  \item Used \cvlanguage{Rust} to implement a physically-based path tracer to produce photorealistic images from a scene description.\textbf{\githublink{prayer}{Project on GitHub.}}
  %    \item Optimized the implementation to take advantage of parallelism as well as acceleration data structures such as KD-Trees.
  %  \end{cvitems}
  %}
  \cventry{Personal Project}
  {\textit{Damage}}
  {}
  {}
  {%
    \begin{cvitems}
    \item Developed a GameBoy emulator for Windows using \cvlanguage{C++}, featuring accurate audio/video emulation and remappable controls.\textbf{\githublink{damage}{Project on GitHub.}}
    \item Created a reusable \cvlanguage{WinAPI} abstraction library to implement the program's user interface.
    \item Reverse-engineered CPU and audio functionality from original developer documentation and other open-source emulators' source code.
    \end{cvitems}
  }
  \cventry{Personal Project}
  {\textit{SSM --- Speedy SIMD Math}}
  {}
  {}
  {%
    \begin{cvitems}
    \item Created a linear algebra library for \cvlanguage{C++} for use in real-time applications such as games.\textbf{\githublink{ssm}{Project on GitHub.}}
    \item Used SIMD intrinsics, coupled with template meta-programming, to implement type-safe yet high-performance matrix, quaternion and vector multiplication routines.
    \item Designed an intuitive, rich API with strong types for unit vectors and quaternions.
    \end{cvitems}
  }
\end{cventries}

\section*{Languages}
\begin{cvitems}
\item \textbf{English: } C2 Level
\item \textbf{Italian: } Native
\item \textbf{French: } B1 Level
\item \textbf{German: } A1 Level
\end{cvitems}
